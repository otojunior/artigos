\section{Implementa{\c c}{\~a}o do Modelo}
\label{sec:implmodelo}
O modelo apresentado foi otimizado para projetos gerenciados pelo Maven. Considerando tal situa{\c c}{\~a}o, {\'e} utilizado dois plugins para automatizar o trabalho de cria{\c c}{\~a}o de branchs e evolu{\c c}{\~a}o de releases. Segue a configura{\c c}{\~a}o dos plugins utilizados que devem ser inseridos no POM do projeto:

\begin{verbatim}
<plugin>
  <groupId>org.codehaus.mojo</groupId>
  <artifactId>build-helper-maven-plugin</artifactId>
  <version>1.12</version>
  <executions>
    <execution>
      <goals>
        <goal>parse-version</goal>
      </goals>
    </execution>
  </executions>
</plugin>

<plugin>
  <groupId>org.apache.maven.plugins</groupId>
  <artifactId>maven-release-plugin</artifactId>
  <version>2.5.3</version>
  <executions>
    <execution>
      <id>dobranch</id>
      <goals>
        <goal>branch</goal>
      </goals>
      <configuration>
        <branchName>
          rel-${parsedVersion.majorVersion}.
          ${parsedVersion.minorVersion}
        </branchName>
        <developmentVersion>
          ${parsedVersion.majorVersion}.
          ${parsedVersion.nextMinorVersion}.
          0-${parsedVersion.qualifier} 
        </developmentVersion>
      </configuration>
    </execution>
    <execution>
      <id>dorelease</id>
      <goals>
        <goal>prepare</goal>
      </goals>
      <configuration>
        <tagNameFormat>v@{project.version}</tagNameFormat>
      </configuration>
    </execution>
  </executions>
</plugin>
\end{verbatim}

Para a cria{\c c}{\~a}o do branch, foi utilizado o comando:

\begin{verbatim}
mvn validate release:branch@dobranch
\end{verbatim}

Para a evolu{\c c}{\~a}o da release, foi utilizado o comando:

\begin{verbatim}
mvn validate release:clean release:prepare@dorelease release:perform release:clean -B
\end{verbatim}

O comando de release acima pode ser usado tamb{\'e}m para a evolu{\c c}{\~a}o de releases no branch master.